\section{Conclusion}
\label{s:concl}
This paper presented the initial design and implementation of \textit{Nail}, a usable interface
generator for data formats. Nail allows programmers to avoid memory corruption and ambiguity
vulnerabilities while reducing effort in parsing and generating real world protocols and file
formats. We have demonstrated Nail's expressive power by implementing DNS and Ethernet. 

Our code is open source on GitHub\footnote{\url{https://github.com/jbangert/nail/}}.

%This paper presented the initial design and implementation of
%\textit{Nail}, a parser generator that aims to minimize the amount
%of programmer effort.  Nail achieves this by reducing the expressive
%power of the grammar, maintaining a \emph{semantic bijection} between
%data formats and internal representations, and allowing programmers to
%specify \emph{structural dependencies} on the data format.  This avoids
%the need for programmers to write explicit semantic actions for input
%and output processing.  We hope that by reducing programmer effort
%required to use a parser generator, Nail will enable more application
%developers to use generated parsers in practice.  Our prototype source
%code is available at \url{https://github.com/jbangert/nail}.

