\documentclass[25pt, a0paper, portrait, margin=0mm, innermargin=15mm,
blockverticalspace=10mm, colspace=15mm, subcolspace=8mm]{tikzposter}

\usepackage{times}
\usepackage{microtype}
\usepackage{booktabs}
\usepackage{mathptmx}
\usepackage{graphicx}
\usepackage{url}
\usepackage{listings}
\usepackage{hyperref}
\usepackage{underscore}
\usepackage[T1]{fontenc}
\usepackage{amsmath}
\usepackage{amsfonts}
\usepackage{amssymb}
\usepackage{verbatim}
\usepackage{color}
\usepackage{xspace}
\usepackage{multirow}
\usepackage{relsize}
\usepackage{fancyvrb}
\usepackage{pgfplots}

\title{Nail: A practical tool for parsing and generating data formats}
\author{Julian Bangert and Nickolai Zeldovich }
\institute{MIT CSAIL}
%\titlegraphic{Logo}
\usetheme{Envelope}

\input{code/fmt}

\begin{document}
\maketitle
\begin{columns}
\column{.5}
\block{Parsing vulnerabilities}{
Current software typically uses hand-implemented parsers, which are not only prone to bugs, but two
separate parser implementations can disagree about the meaning of an input. This has been the source
of many recent high-profile vulnerabilities:
\begin{itemize}
\item Evasi0n jailbreaks on iOS~\cite{evaders6}.
\item PKI layer cake~\cite{Marlinspike:SSL},\cite{DBLP:conf/fc/KaminskyPS10}.
\item Android master key ~\cite{saurik-masterkey}
\end{itemize}

{\bf Case study of ZIP vulnerabilities in the CVE database:}

\begin{tabular}{lllr}
\toprule
 \bf Classification  & \bf Example description & \bf Count\\
\midrule
  Memory corruption  & Buffer overflow & 11\\
  Parsing inconsistency & Virus scanners interpret ZIP files incorrectly & 4\\
  Semantic misunderstanding & Weak cryptography used even if user selects AES & 1\\
\midrule
  \multicolumn{2}{l}{\bf Total of all vulnerabilities related to .zip processing }& \bf 16  \\
\bottomrule
  
\end{tabular}
}
\block{Solution: Grammar-based Parsers}{   
Grammar-based parsers, such as \texttt{bison} or Hammer~\cite{hammer-parser}, are compact, reuseable
and safer. However, traditional parsers are inconvenient to use, because the programmer has to write all the
yellow-colored parts:

\includegraphics{StateOfTheArt.png}

Nail grammars describe both the format and a data type to represent it. 

\includegraphics{Nail.png}
  
}
\block{Complicated real-world example: ZIP }{
  \includegraphics[width=0.4\columnwidth]{ZipHorizontal.png}
  \begin{itemize}
    \item Variable-length end-of-file header
    \item Length and offset fields
    \item Duplicated information
    \item Checksums, compression and encryption
  \end{itemize}
}
\block{Design Goals}{
  \begin{itemize}
    \item Reduce programmer effort
  \begin{itemize}
      \item Semantic combinators describes data type and format.
      \item Generate output.
  \end{itemize}
     \item Handle complex formats
  \begin{itemize}
       \item Dependency Fields: hide redundant information.
       \item Transforms: Extensibility.
       \end{itemize}
  \end{itemize}
}
\block{Semantic Bijection}{
  Traditional bijections don't make sense for some data formats. For example, offsets are and should
  be discarded during parsing. Nail parsers and generators therefore form a looser 'semantic
  bijection', in which parser(generator(x)) = x, but not necessarily vice versa.

}
\column{.5}
\block{Nail Grammar for DNS}{
\small
\begin{Verbatim}[commandchars=\\\{\},codes={\catcode`\$=3\catcode`\^=7\catcode`\_=8}]
\PY{n}{zip} \PY{o}{=} \PY{p}{\PYZob{}} 
 \PY{err}{\PYZdl{}}\PY{n}{filestream}\PY{p}{,} \PY{err}{\PYZdl{}}\PY{n}{end\PYZus{}directory} \PY{n}{transform} \PY{n}{zip\PYZus{}eod} \PY{p}{(}\PY{err}{\PYZdl{}}\PY{n}{current}\PY{p}{)}
 \PY{n}{contents} \PY{n}{apply} \PY{err}{\PYZdl{}}\PY{n}{end\PYZus{}directory} \PY{n}{end\PYZus{}of\PYZus{}directory}\PY{p}{(}\PY{err}{\PYZdl{}}\PY{n}{filestream}\PY{p}{)}
\PY{p}{\PYZcb{}}
\PY{n}{end\PYZus{}of\PYZus{}directory}\PY{p}{(}\PY{err}{\PYZdl{}}\PY{n}{filestream}\PY{p}{)} \PY{o}{=} \PY{p}{\PYZob{}} 
 \PY{n}{uint32} \PY{o}{=} \PY{l+m+mh}{0x06054b50}
 \PY{n}{disks} \PY{n}{uint16} \PY{o}{|} \PY{p}{[}\PY{l+m+mi}{0}\PY{p}{]}
 \PY{n}{directory\PYZus{}disk} \PY{n}{uint16} \PY{o}{|} \PY{p}{[}\PY{l+m+mi}{0}\PY{p}{]}
 \PY{err}{@}\PY{n}{this\PYZus{}records} \PY{n}{uint16}
 \PY{err}{@}\PY{n}{total\PYZus{}records} \PY{n}{uint16} 
 \PY{n}{transform} \PY{n}{uint16\PYZus{}depend} \PY{p}{(}\PY{err}{@}\PY{n}{this\PYZus{}records} \PY{err}{@}\PY{n}{total\PYZus{}records}\PY{p}{)}
 \PY{err}{@}\PY{n}{directory\PYZus{}size} \PY{n}{uint32} 
 \PY{err}{@}\PY{n}{directory\PYZus{}start} \PY{n}{uint32}
 \PY{err}{\PYZdl{}}\PY{n}{dirstr1} \PY{n}{transform} \PY{n}{offset\PYZus{}u32} \PY{p}{(}\PY{err}{\PYZdl{}}\PY{n}{filestream} \PY{err}{@}\PY{n}{directory\PYZus{}start}\PY{p}{)} 
 \PY{err}{\PYZdl{}}\PY{n}{directory\PYZus{}stream} \PY{n}{transform} \PY{n}{size\PYZus{}u32} \PY{p}{(}\PY{err}{\PYZdl{}}\PY{n}{dirstr1} \PY{err}{@}\PY{n}{directory\PYZus{}size}\PY{p}{)}
 \PY{err}{@}\PY{n}{comment\PYZus{}length} \PY{n}{uint16}
 \PY{n}{comment} \PY{n}{n\PYZus{}of} \PY{err}{@}\PY{n}{comment\PYZus{}length} \PY{n}{uint8}
 \PY{n}{files} \PY{n}{apply} \PY{err}{\PYZdl{}}\PY{n}{directory\PYZus{}stream} \PY{n}{n\PYZus{}of} \PY{err}{@}\PY{n}{total\PYZus{}directory\PYZus{}records} \PY{n}{dir\PYZus{}fileheader}\PY{p}{(}\PY{err}{\PYZdl{}}\PY{n}{filestream}\PY{p}{)}
\PY{p}{\PYZcb{}}
\PY{n}{dir\PYZus{}fileheader}\PY{p}{(}\PY{err}{\PYZdl{}}\PY{n}{filestream}\PY{p}{)} \PY{o}{=} \PY{p}{\PYZob{}}  \PY{c+cm}{/*cut for brevity*/}
 \PY{n}{uint32} \PY{o}{=} \PY{l+m+mh}{0x02014b50}\PY{c+cm}{/*...*/}
 \PY{err}{@}\PY{n}{compressed\PYZus{}size} \PY{n}{uint16} \PY{c+cm}{/*...*/}
 \PY{err}{@}\PY{n}{crc32} \PY{n}{uint32}\PY{c+cm}{/*...*/}
 \PY{err}{@}\PY{n}{file\PYZus{}name\PYZus{}len} \PY{n}{uint16}\PY{c+cm}{/*...*/}
 \PY{err}{@}\PY{n}{off} \PY{n}{uint32}
 \PY{n}{filename} \PY{n}{n\PYZus{}of} \PY{err}{@}\PY{n}{file\PYZus{}name\PYZus{}len} \PY{n}{uint8}\PY{c+cm}{/*...*/}
 \PY{err}{\PYZdl{}}\PY{n}{contentstream} \PY{n}{transform} \PY{n}{offset\PYZus{}u32} \PY{p}{(}\PY{err}{\PYZdl{}}\PY{n}{filestream} \PY{err}{@}\PY{n}{off}\PY{p}{)}
 \PY{n}{contents} \PY{n}{apply} \PY{err}{\PYZdl{}}\PY{n}{contentstream} \PY{n}{fileentry}\PY{p}{(}\PY{err}{@}\PY{n}{crc32}\PY{p}{,}\PY{err}{@}\PY{n}{compressed\PYZus{}size} \PY{c+cm}{/*...*/}\PY{p}{)}
\PY{p}{\PYZcb{}}
\PY{n}{fileentry}\PY{p}{(}\PY{err}{@}\PY{n}{crc32} \PY{n}{uint32}\PY{p}{,}\PY{err}{@}\PY{n}{size} \PY{n}{uint32}\PY{c+cm}{/*...*/}\PY{p}{)} \PY{o}{=} \PY{p}{\PYZob{}}
 \PY{n}{uint32} \PY{o}{=} \PY{l+m+mh}{0x04034b50}\PY{c+cm}{/*...*/}
 \PY{err}{@}\PY{n}{crc\PYZus{}local} \PY{n}{uint32}\PY{c+cm}{/*...*/}
 \PY{err}{\PYZdl{}}\PY{n}{compressed} \PY{n}{transform} \PY{n}{size\PYZus{}u32} \PY{p}{(}\PY{err}{\PYZdl{}}\PY{n}{current} \PY{err}{@}\PY{n}{size}\PY{p}{)}
 \PY{err}{\PYZdl{}}\PY{n}{uncompressed} \PY{n}{transform} \PY{n}{zip\PYZus{}compression} \PY{p}{(}\PY{err}{\PYZdl{}}\PY{n}{compressed} \PY{err}{@}\PY{n}{size}\PY{c+cm}{/*...*/}\PY{p}{)}
 \PY{n}{transform} \PY{n}{crc\PYZus{}32} \PY{p}{(}\PY{err}{\PYZdl{}}\PY{n}{uncompressed} \PY{err}{@}\PY{n}{crc32}\PY{p}{)}
 \PY{n}{contents} \PY{n}{apply} \PY{err}{\PYZdl{}}\PY{n}{uncompressed} \PY{n}{many} \PY{n}{uint8}
 
 \PY{n}{transform} \PY{n}{uint16\PYZus{}depend} \PY{p}{(}\PY{err}{@}\PY{n}{crc\PYZus{}local} \PY{err}{@}\PY{n}{crc32}\PY{p}{)}\PY{c+cm}{/*...*/}
\PY{p}{\PYZcb{}}
\end{Verbatim}

}
\block{Dependent Fields}{
Duplicated data and layout information can cause parser consistency bugs. Nail represents them with
Dependent Fields, which are automatically verified during parsing and created during 
generation. 
}
\block{Streams and Transformations}{
  Existing parsers are linear, consuming input front to back; Offset fields require kludges.
  Nail grammars feature multiple streams, and introduce \emph{Transformations} to create
  new streams. Transformations are a pair of functions operating on streams and dependent fields.
  A standard library of Transformations is provided, but for complex formats, programmers can write
  their own Transformations, for which they have to guarantee security and semantic bijection.
 For example, in the ZIP grammar above, the programmer has to write two functions to find and
 generate the end-of-file header, with the following prototype:
\begin{Verbatim}[commandchars=\\\{\},codes={\catcode`\$=3\catcode`\^=7\catcode`\_=8}]
\PY{k+kt}{int} \PY{n+nf}{dnscompress\PYZus{}parse}\PY{p}{(}\PY{n}{NailArena} \PY{o}{*}\PY{n}{tmp}\PY{p}{,}
  \PY{n}{NailStream} \PY{o}{*}\PY{n}{out\PYZus{}decomp}\PY{p}{,}
  \PY{n}{NailStream} \PY{o}{*}\PY{n}{in\PYZus{}current}\PY{p}{)}\PY{p}{;}

\PY{k+kt}{int} \PY{n+nf}{dnscompress\PYZus{}generate}\PY{p}{(}\PY{n}{NailArena} \PY{o}{*}\PY{n}{tmp}\PY{p}{,}
  \PY{n}{NailStream} \PY{o}{*}\PY{n}{in\PYZus{}decomp}\PY{p}{,}
  \PY{n}{NailStream} \PY{o}{*}\PY{n}{out\PYZus{}current}\PY{p}{)}\PY{p}{;}
\end{Verbatim}

}
\block{Related Work and Bibliography}{
  \begin{itemize}
    \item Interface generators such XDR~\cite{RFC:1832} and ProtoBuf~\cite{varda2008} can parse and
      generate, but cannot interoperate with legacy formats.
    \item Data Description Languages, such as PacketTypes~\cite{mccann2000packet},
      DataScript~\cite{back2002datascript} and PADS~\cite{Fisher:2005:PDL:1064978.1065046}, can
      parse and generate, but are not extensible, so they cannot support complicated encodings in
      formats like ZIP.
    \item See our paper for a detailed discussion, including application use of parsers.
  \end{itemize}
\begin{minipage}{\textwidth}
\tiny \bibliography{n-str,main,locasto,langsec,n,n-conf}
\bibliographystyle{alpha} 
\end{minipage}
}
  
\end{columns}
\end{document}