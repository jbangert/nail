\section{Evaluation}
\label{s:eval}

\begin{figure*}

\label{fig:ethernet}
\end{figure*}

In our evaluation of Nail, we answer four questions:

\begin{itemize}

\item Can Nail grammars support real-world data formats, and
      are Nail's techniques critical to handling these formats?

\item How much programmer effort is required to build an
      application that uses Nail for data input and output?

\item Does using Nail for handling input and output improve
      application security?

\item Does Nail achieve acceptable performance?

\end{itemize}

\subsection{Data formats}
\label{s:eval-formats}

To answer the first question, we used Nail to implement grammars
for seven protocols with a range of challenging features.
Figure~\ref{fig:eval-protocols} summarizes these protocols, the lines
of code for their Nail grammars, and the challenging features that make
the protocols difficult to parse with state-of-the-art parser generators.
We find that despite the challenging aspects of these protocols, Nail is
able to capture the protocols, by relying on its novel features: dependent
fields, streams, and transforms.  In contrast, state-of-the-art parser
generators would be unable to fully handle 5 out of the 7 data formats.
In the rest of this subsection, we describe the DNS and Zip grammars in
more detail, focusing on how Nail's features enable us to support these
formats.

\begin{figure}[tb]
\centering
\begin{tabular}{lrl}
\toprule
\textbf{Protocol} & \textbf{LoC} & \textbf{Challenging features} \\ 
\midrule
DNS packets & 48+64 & Label compression,\\
  & & count fields \\
ZIP archives & 92+78 & Checksums, offsets, \\ 
  & & variable length trailer, \\
  & & compression \\
Ethernet  & 16+0\phantom{0} & --- \\
ARP       & 10+0\phantom{0} & --- \\
IP        & 25+0\phantom{0} & Total length field, options \\
UDP       &  7+0\phantom{0} & Checksum, length field \\
ICMP      &  5+0\phantom{0} & Checksum \\
\bottomrule
\end{tabular}

\caption{Protocols, sizes of their Nail grammars, and challenging aspects
of the protocol that cannot be expressed in existing grammar languages.
A + symbol counts lines of Nail grammar code (before the +) and lines of
C code for protocol-specific transforms (after the +).}
\label{fig:eval-protocols}
\end{figure}

\paragraph{DNS.}

Figure~\ref{fig:dns-xform} presents the Nail grammar for DNS packets.
The grammar corresponds almost directly to the diagrams in RFC 1035,
which defines DNS~\cite[\S4]{RFC:1035}.  Each DNS packet consists of a
header, a set of question records, and a set of answer records. Domain
names in both queries and answers are encoded as a sequence of labels,
terminated by a zero byte.  Labels are Pascal-style strings, consisting
of a length field followed by that many bytes comprising the label.

One challenging aspect of DNS packets lies in the count fields (\cc{qc},
\cc{ac}, \cc{ns}, and \cc{ar}), which represent the number of questions
or answers in another part of the packet.  Nail's \cc{n_of} combinator
handles this situation easily, which would have been difficult to handle
for other parsers.

Another challenging aspect of DNS is label
compression~\cite[\S4.1.4]{RFC:1035}.  Label compression is used to reduce
the size overhead of including each domain name multiple times in a DNS
reply (once in the question section, and at least once in the response
section).  If a domain name suffix is repeated, instead of repeating that
suffix, the DNS packet may write a two-bit marker sequence followed by
a 14-bit offset into the packet, indicating the position of where that
suffix was previously encoded.

\begin{figure}
\begin{verbatim}
labels = <many { @length uint8 | 1..255
                 label n_of @length uint8 }
          uint8 = 0>
compressed ={
  $decompressed transform dnscompress($current);
  labels apply $decompressed labels
}
question = {
  labels compressed_lbls
  qtype uint16 | 1..16
  qclass uint16 | [1,255]
}
answer = {
  labels labels
  rtype uint16 | 1..16
  class uint16 | [1]
  ttl uint32
  @rlength uint16
  rdata n_of @rlength uint8
}
dnspacket = {
  id uint16
  qr uint1
  opcode uint4
  aa uint1
  tc uint1
  rd uint1
  ra uint1
  uint3 = 0
  rcode uint4
  @qc uint16
  @ac uint16
  uint16 = 0    // authority
  uint16 = 0    // additional
  // We don't support authority or
  // additional sections in the prototype
  questions n_of @qc question
  responses n_of @ac answer
}
\end{verbatim}
\caption{Nail grammar for DNS packets, used by our prototype DNS server.}
\label{fig:dns-full}
\end{figure}


Handling label compression in existing tools, such as Bison or Hammer,
would at best be very awkward, because some ad-hoc trick would have to
be used to re-position the parser's input stream.  Keeping track of the
position of all recognized labels would not be enough, as the offset field
may refer to any byte within the packet, not just the beginning of labels.
For this reason, the DNS server used as the example for Hammer does not
support compression.

\begin{figure}
\smaller[0.5]
\begin{Verbatim}[commandchars=\\\{\},codes={\catcode`\$=3\catcode`\^=7\catcode`\_=8}]
\PY{k+kt}{int} \PY{n+nf}{dnscompress\PYZus{}parse}\PY{p}{(}\PY{n}{NailArena} \PY{o}{*}\PY{n}{tmp}\PY{p}{,}
  \PY{n}{NailStream} \PY{o}{*}\PY{n}{out\PYZus{}decomp}\PY{p}{,}
  \PY{n}{NailStream} \PY{o}{*}\PY{n}{in\PYZus{}current}\PY{p}{)}\PY{p}{;}

\PY{k+kt}{int} \PY{n+nf}{dnscompress\PYZus{}generate}\PY{p}{(}\PY{n}{NailArena} \PY{o}{*}\PY{n}{tmp}\PY{p}{,}
  \PY{n}{NailStream} \PY{o}{*}\PY{n}{in\PYZus{}decomp}\PY{p}{,}
  \PY{n}{NailStream} \PY{o}{*}\PY{n}{out\PYZus{}current}\PY{p}{)}\PY{p}{;}
\end{Verbatim}

\caption{Signatures of stream transform functions for handling DNS label compression.}
\label{fig:dns-xform}
\end{figure}

In contrast, Nail is able to handle label compression, by using a stream
transform; the signatures of the two transform functions are shown
in Figure~\ref{fig:dns-xform}.  When parsing a packet, this transform
decompresses the DNS label stream by following the offset pointers.
When generating a packet, this transform receives the current suffix as
an input, and scans the packet so far for previous occurrences, which
implements compression.


\paragraph{ZIP files.}

An especially tricky data format is the ZIP compressed archive
format~\cite{pkzip}.  ZIP files are normally parsed end-to-beginning. At
the end of each ZIP file is an \emph{end-of-directory header}. This header
contains a variable-length comment, so it has to be located by scanning
backwards from the end of the file until a magic number and a valid
length field is found. Many ZIP implementations disagree on how to find
this header in confusing situations, such as when the comment contains
the magic number~\cite{wolf:berlinsides-zip}.  This end-of-directory
header contains the offset and size of the \emph{ZIP directory}, which
is an array of \emph{directory entry headers}, one for every file
in the archive.  Each entry stores file metadata, such as file name,
compressed and uncompressed size, and a checksum, in addition to the
offset of a \emph{local file header}. The local file header duplicates
most information from the directory entry header and the compressed file
contents follow it immediately.

\begin{figure}
\smaller[0.5]
\begin{verbatim}
zip_file = { 
 $file, $header transform 
    zip_end_of_directory ($current)
  contents apply $header
    end_of_directory($file)
}
end_of_directory($file) = {//[...]
 @directory_size uint32 
 @directory_start uint32
 $dirstr1 transform offset_u32 
     ($filestream @directory_start) 
 $directory_stream transform size_u32 
     ($dirstr1 @directory_size)
 @comment_length uint16
 comment n_of @comment_length uint8
 files apply $directory_stream n_of 
     @total_directory_records dir_entry($file)
}
dir_entry($file) = {
//[...]  
  @compression_method uint16      
  mtime uint16
  mdate uint16
  @crc32 uint32
  @compressed_size uint32
  @uncompressed_size uint32
  @file_name_len uint16
  @extra_len uint16
  @comment_len uint16//[...]
  @off uint32
  filename n_of @file_name_len uint8
  extra_field n_of @extra_len uint8
  comment n_of @comment_len uint8
  $content transform offset_u32 ($file @off)
  contents apply $content file(@crc32,
     @compression_method,@compressed_size, 
     @uncompressed_size)
}
file(@crc32 uint32, @method uint16,
 @compressed_size uint32, 
 @uncompressed_size uint32) = { 
  uint32 = 0x04034b50
  version uint16
  flags file_flags
  @method_lcl uint16//[...]
  $compressed transform size_u32 
           ($current @compressed_size)
  $uncompressed transform zip_compression 
           ($compressed @method )
  transform crc_32 ($uncompressed @crc32)
  contents apply $uncompressed many uint8
  transform u16_depend (@method_lcl @method)
  //[...]
}
\end{verbatim}
\caption{Nail grammar for ZIP files. Various fields have been cut for brevity.}
\label{fig:zip-extract}
\end{figure}


Duplicating information made sense when ZIP files were stored on floppy
disks with slow seek times and high fault rates, and memory constraints
made it impossible to keep the ZIP directory in memory or the archive
was split across multiple disks.  However, care must be taken that the
metadata is consistent. For example, vulnerabilities could occur if
the length in the central directory is used to allocate memory and the
length in the local directory is used to extract without checking that
they are equal first, as was the case in the Python ZIP
library~\cite{cve-python-zip-maybe}.

\begin{figure}[tb]
\smaller[0.5]
\begin{Verbatim}[commandchars=\\\{\},codes={\catcode`\$=3\catcode`\^=7\catcode`\_=8}]
\PY{k+kt}{int} \PY{n+nf}{zip\PYZus{}end\PYZus{}of\PYZus{}directory\PYZus{}parse}\PY{p}{(}
  \PY{n}{NailArena} \PY{o}{*}\PY{n}{tmp}\PY{p}{,} \PY{n}{NailStream} \PY{o}{*}\PY{n}{out\PYZus{}files}\PY{p}{,}
  \PY{n}{NailStream} \PY{o}{*}\PY{n}{out\PYZus{}dir}\PY{p}{,} \PY{n}{NailStream} \PY{o}{*}\PY{n}{in\PYZus{}current}\PY{p}{)}\PY{p}{;}
\PY{k+kt}{int} \PY{n+nf}{zip\PYZus{}end\PYZus{}of\PYZus{}directory\PYZus{}generate}\PY{p}{(}
  \PY{n}{NailArena} \PY{o}{*}\PY{n}{tmp}\PY{p}{,} \PY{n}{NailStream} \PY{o}{*}\PY{n}{in\PYZus{}files}\PY{p}{,}
  \PY{n}{NailStream} \PY{o}{*}\PY{n}{in\PYZus{}dir}\PY{p}{,} \PY{n}{NailStream} \PY{o}{*}\PY{n}{out\PYZus{}current}\PY{p}{)}\PY{p}{;}
\end{Verbatim}

\caption{Signatures of stream transform functions for handling the
end-to-beginning structure of ZIP files.}
\label{fig:zip-eod-xform}
\end{figure}

Figure~\ref{fig:zip-extract} shows an abbreviated version of our ZIP
file grammar.  The ZIP grammar is a good example of how transformations
capture complicated syntax in a real-world file format; existing parser
languages cannot handle a file format of this complexity.

The \cc{zip\_file} grammar first splits the entire file stream into two
streams based on the \cc{zip\_end\_of\_directory} transform (whose two
C function signatures are shown in Figure~\ref{fig:zip-eod-xform}).
\cc{zip\_end\_of\_directory_parse} finds the end-of-directory header
as described above, by scanning the file backwards, and splits the
file into two streams, one containing the end-of-directory header and
one containing the file contents.  The header stream is then parsed,
according to the \cc{apply} rule, using the \cc{end\_of\_directory}
parser.  The file stream, to which all the offsets in the metadata refer,
is passed as an argument to that parser.  The directory header is then
parsed, and the offset and size transforms provided by Nail are used
to isolate the actual directory from the file stream.  Each directory
entry in turn points to a local file header.

The \cc{file} grammar for a ZIP file entry is particularly interesting.
It receives as arguments the file size, checksum, and compression method
of the file from the directory header.  However, this same information
is duplicated in the file entry, so the grammar uses the Nail-supplied
\cc{depend} transform to check whether the two values are equal. Unlike
most other transforms, \cc{depend} does not consume or produce strings; it
only checks that two dependent fields are equal when parsing, and assigns
the value of the second field to the first when generating.  This ensures
that the programmer does not have to worry about inconsistencies when
handling the internal representation of a ZIP file.

Immediately following the file entry is the compressed data.  Because
most compression algorithms operate on unbounded streams of data, Nail
decompresses data in two steps.  First, it isolates the compressed
data from the rest of the stream by using the \cc{size} transform,
which operates on the current stream, meaning it will consume data
starting at the current position of the parser in the input.  Second,
Nail invokes a custom \cc{zip\_compression} transform that implements
the appropriate compression and decompression functions based on the
specified compression method. These functions are otherwise oblivious
to the layout or metadata of the file.


\subsection{Programmer effort}


%code size unzip 6.0
%2821 lines
\label{s:eval-effort}
To answer the second question, we implemented small example applications based on the above
grammars and compared code size with comparable applications that process data manually. We will
also compare a toy DNS server implemented in Nail with a similar toy DNS server provided as an
example for the Hammer parser generator. All code size measurements for C were performed by
SlocCount\cite{sloccount}.

\noindent We implemented the following applications and compare them to applications written without
Nail:

\noindent\begin{tabular}{@{}lll@{}} 
\toprule
\textbf{System} & \textbf{Nail LoC}  & \textbf{LoC w/o Nail} \\
\midrule
DNS server & 295 & 683 (Hammer parser)\\
&&  ~10.000 (DJBDNS)\\ 
DNS client & 210  & 4,615 (host) \\
\texttt{unzip} & 220 & 1600 (Info-Zip) \\
\bottomrule
\end{tabular}
%%DNS: 
\paragraph{DNS.}

Our DNS grammar, which is partly reproduced in Figure~\ref{fig:dns-full}, consists of a 48 line Nail
grammar and 64 lines of C implementing DNS label compression, whose signatures are shown in
Figure~\ref{fig:dns-xform}.
The grammar describes both the structure of DNS packets (36 lines) and of a simple zone-file format
(16 lines) supporting A, NS, MX and CNAME records. 

  
With this grammar, we built a simple authoritative DNS server, which parses a zone file, listens to incoming DNS
requests, parses them and generates appropriate responses, is implemented in 183 lines of C. The
same grammar is used, together with 98 lines of C, to implement a minimal clone of the
\cc{host} command-line tool. Most of this code consists of a custom hashtable and system
interface code, such as listening to sockets and reading the zonefile. 

The Hammer parser framework\cite{hammer-parser} ships with a toy DNS server that responds to
any valid DNS query with a CNAME record to the domain ``spargelze.it''. 
The server consists of 683 lines of C, mostly custom validators, semantic actions,
and data structure definitions, with only 52 lines of code defining the
grammar with Hammer's combinators.

It is hard to compare the programming effort required to implement
our DNS server to that of a real world DNS server, since we implement less functionality.
 In particular, we do not send the additional glue records real-world DNS servers send and
we did not implement any configuration options.

However, the closest in functionality and intent is Dan Bernstein's
djbdns,\footnote{\url{http://cr.yp.to/djbdns.html}} which aims to be a minimalist, highly secure,
authoritative-only DNS server. The latest release of djbdns, including various support tools, is
about 10,000 lines of C as measured by \cc{sloccount}. DJ

\paragraph{ZIP.}


We wrote a grammar for ZIP, which consists of 92 lines of Nail and 78 lines of C implementing two
stream transforms, one for the DEFLATE compression algorithm with the help of the zlib library and
one for finding the end-of-directory header. 

Our grammar is also very efficient to use. Using our grammar, we build a ZIP file extractor in 50
lines of C. Because more recent versions of ZIP have added more features, such as large file support
and encryption, the closest existing tool in functionality is the historic version 5.4 of the Info-Zip unzip
utility\cite{infozip} that is shipped with most Linux distributions. The entire unzip distribution
is about 46.000 lines of code, which is mostly optimized implementation of various compression
algorithms and other configuration and portability code.

However, unzip isolates the equivalent of our Nail tool in the file extract.c, which parses the ZIP
metadata and calls various decompression routines in other files. This file measures over 1,600
lines of C. 

\subsection{Security.} 
We use a twofold approach to evaluate the security of applications implemented with Nail. First, we
analyze a list of CVE's related to the ZIP file format and argue how our ZIP tools based on Nail are
immune against those vulnerability classes. Second, we present the results of fuzz-testing our DNS
server.

The ZIP format has been associated with many vulnerabilities, and the PROTOS Genome project found numerous
security vulnerabilities in most implementations of ZIP and other archive formats that are directly
related to input handling.

Between May 2011 and May 2014, the CVE database\cite{cve-database} lists 83 vulnerabilities when searching for the
string ``ZIP''. Of these, 42 were completely unrelated to the archive format, for example reporting
vulnerabilities revealing Zip codes of other users. 26 vulnerabilities were related to ZIP archives,
but not to input processing, such as trusting arbitrary update ZIP files sent to a server or
vulnerabilities that can be accessed through manipulating files in a ZIP container. This includes
several vulnerabilities allowing another files to replaced while they are extracted from ZIP
files because insufficiently secure temporary directories are used. Of the 15 remaining
vulnerabilities pertaining to ZIP files, 14 vulnerabilities were related to input processing. The
 other vulnerability, which falls outside of Nails purview, concerns a ZIP packer choosing a weak
 legacy cryptosystem even when configured to use AES.

 Nail parsers are immune by design to vulnerabilities like these 14. Ten of those vulnerabilities
 are memory corruption attacks, that could also be avoided by using safer languages or better
 verification tools. Most of the buffer overflows seem to result from inconsistent offset fields
 being manipulated without checking. Because Nail's generated code checks offsets before reading and
 does not expose any untrusted pointers to the application, it protects against memory corruption
 attacks by design.

The remaining four vulnerabilities do not involve memory corruption, but rather parser ambiguities.
Two of them relate to implementations mis-parsing ZIP files, particularly in virus scanners. One
other vulnerability is in a Python ZIP library, which uses both the size field in the file entry and
in the header, without checking for their consistency. This results in a denial-of-service through
endless loops or incorrectly extracted ZIP files.

A similar attack was used by the fourth vulnerability, the infamous Android master key bug
\cite{saurik-masterkey}, that completely by-passed Android security multiple times through parser
differentials between the ZIP handler that checks signatures for privileged applications and the ZIP
implementations that ultimately extracts those files. Thus, valid  application bundles could be
modified to include malicious applications without breaking their signatures. Instead of both
components containing an ad-hoc archive implementation, a single Nail grammar would be reusable
across both implementations. 

To provide additional assurance that Nail parsers are free of memory corruption attacks, we ran the
DNS fuzzer provided with the Metasploit\cite{mspframework} framework on our server, which sent
randomly corrupted DNS queries to our server for 4 hours, during which it did not crash or trigger
the stack or heap corruption detector.
\subsection{Performance.}
\begin{figure}
\begin{tikzpicture}
  \begin{axis}[
      width=0.9\textwidth,
  y=0.75cm,
  ytick={1,2},
  yticklabels={NailDNS, Bind 9},
  xtick={0,50000,100000,150000,200000},
  xmin=0,
  xlabel ={Throughput},
  x unit ={\frac{\text{queries}}{s}},
  scaled x ticks = false,
  x tick label style = {/pgf/number format/fixed}
]
\addplot+[boxplot]
  table[row sep=\\,y index=0] {
    data\\
    190733.210721\\
189364.319996\\
146137.368999\\
191168.613247\\
187743.527772\\
191879.571775\\
194041.878838\\
 };
\addplot+[boxplot]
  table[row sep=\\,y index=0] {
data\\
50713.068173\\
45230.492090\\
47824.395302\\
39167.055247\\
47234.594953\\
59239.056531\\
40436.939492\\
 };
\end{axis}
\end{tikzpicture}

\begin{tikzpicture}
\begin{axis}[
    y=0.75cm,
    width=0.9\textwidth,
  ytick={1,2},
  yticklabels={NailDNS, Bind 9},
  xmin=0,
  xlabel ={Round Trip Time},
  x unit= {msec},
  xtick={0, 0.1, 0.2,0.3,0.4,0.5},
  scaled x ticks = false
]
\addplot+[boxplot]
table[row sep=\\,y index=0] {
data\\
0.097\\
0.097\\
0.125\\
0.096\\
0.098\\
0.095\\
0.094\\
 };
\addplot+[boxplot]
  table[row sep=\\,y index=0] {
data\\
0.388\\
0.434\\
0.411\\
0.501\\
0.416\\
0.333\\
0.486\\
 };
\end{axis}
\end{tikzpicture}

\caption{A box plot comparing the performance of the Nail-based DNS server compared to
  BIND 9.5.5 on 50,000 domains.  The boxes show the interquartile range, with the middle showing the
  median result. The dots show outliers.}
\label{fig:perf-dns}

\end{figure}

To estimate the performance impact of using Nail grammars as opposed to hand-written grammars, we
benchmarked our DNS server against BIND.

We compare the performance of our DNS server to the release 9.9.5 of ISC BIND 9\cite{bind9}, one of
the oldest and most popular DNS daemons, on a load that resembles the authoritative name server at a
hosting company. First, we generated domain names consisting of one or
two labels randomly selected from an English dictionary and one label that is one of three popular
Top Level Domains: ``com'', ``net'' or ``org''. Second, we randomly selected 90\% of these domains and
created a zone file that mapped these domain names to the local IP address. 

Finally, we used the \cc{queryperf} tool provided with BIND to query each domain between zero
and three times on a local instance of BIND or NailDNS. Both DNS servers were operating on a single, 
core  of an Intel i7-3610QM with 12GB of RAM on an idle system, and the benchmark tool kept at most 20
queries outstanding at once. The queryperf tool was configured to
repeat the same, randomized sequence of queries for one minute and the average throughput and
response latency were measured.  We repeated each test seven times  with 50.000
domain names, restarting each daemon in between. We also performed one initial dry run to warm the
file system cache for the zone file. We repeated the tests with 1 million domain names and found
that the servers performed almost identically.

Surprisingly, our simple Nail server outperforms the established BIND server that has seen decades
of optimization by a factor of 3.  This demonstrates that even though Nail was
not developed focusing on performance, applications built on Nail can compete with real-world
hand-written parsers.



